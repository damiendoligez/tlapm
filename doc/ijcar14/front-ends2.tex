\documentclass[11pt,fleqn]{article}
\listfiles
\usepackage{latexsym}
\usepackage{tla2}


\newcommand{\dd}{\,\ldotp\ldotp}
\newcommand{\tth}{\ensuremath{^{\rm th}}}
\newcommand{\st}{\ensuremath{^{\rm st}}}
\newcommand{\tlaplus}{TLA$^{+}$}
\newcommand{\ov}[1]{\ensuremath{\overline{\rule{0pt}{.65em}#1}}}
\newcommand{\en}{{\rm\textsc{enabled}}}
\newcommand{\enb}{{\textit{Enabled}}}
\newcommand{\M}[1]{\ensuremath{[\![#1]\!]}}
\newcommand{\MM}[2]{\ensuremath{[\![#1]\!](#2)}}
\newcommand{\B}{\begingroup\tlachars\BB}
\newcommand{\BB}[1]{\endgroup\,\fbox{\rule{0pt}{.7em}%
\let\tlacolon\midcolon\ensuremath{#1}}\,}
\newcommand{\C}[1]{\ensuremath{\mathcal{C}\{#1\}}}
\newcommand{\EN}[1]{\ensuremath{\mathcal{E}\{\!|#1]\!\}}}
\renewcommand{\X}[1]{\ensuremath{\mathcal{X}\{\!|#1]\!\}}}
\renewcommand{\WITH}{\;\textsc{with}\;}
\renewcommand{\O}{\C{O, e_{NL}}}
\renewcommand{\L}[2]{\ensuremath{\mathcal{L}\{\![#1,#2]\!\}}}
\newcommand{\Lz}[1]{\ensuremath{\mathcal{L}\{\![#1]\!\}}}
%\newcommand{\Sj}[2]{\ensuremath{(\!(#1,#2)\!)}}
\renewcommand{\P}[1]{\ensuremath{\mathcal{P}\{\!|#1]\!\}}}
\newcommand{\oneto}[2]{\ensuremath{#1_{1},\,\ldots,\,#1_{#2}}}
\newtheorem{observation}{Observation}
\newtheorem{conjecture}{Conjecture}
\newtheorem{theorem}{Theorem}

%&c&\textsc{#}&
%&b&\textbf{#}&

\title{\textbf{The \tlaplus\ Front-Ends}}
\author{The TLAPS Project}
\date{\moddate}

%%%%%%%%%%%%%%%%%%%%%%%%%%%%%%%%%%%%%%%%%%%%%%%%%%%%%%%%%%%%%%%%%%%
%                 MODIFICATION DATE                               %
%%%%%%%%%%%%%%%%%%%%%%%%%%%%%%%%%%%%%%%%%%%%%%%%%%%%%%%%%%%%%%%%%%%
%                                                                 %
% Defines \moddate to expand to modification date such as         %
%                                                                 %
%    5 Aug 1991                                                   %
%                                                                 %
% and \prdate to print it in a large box.  Assumes editor         %
% updates modification date in standard SRC Gnu Emacs style.      %
% (should work for any user name).                                %
%%%%%%%%%%%%%%%%%%%%%%%%%%%%%%%%%%%%%%%%%%%%%%%%%%%%%%%%%%%%%%%%%%%
\def\ypmd{%                                                       %
%                                                                 %
%                                                                 %
  Last modified on Sun  1 December 2013 at 18:22:30 PST by lamport        %
  endypmd}                                                        %
%                                                                  %
%%%%%%%%%%%%%%%%%%%%%%%%%%%%%%%%%%%%%%%%%%%%%%%%%%%%%%%%%%%%%%%%%%%
\newcommand{\moddate}{\expandafter\xpmd\ypmd}                     %
\def\xpmd Last modified                                           %
on #1 #2 #3 #4 at #5:#6:#7 #8 by #9 endypmd{#2 #3 #4}                %
\newcommand{\prdate}{\noindent\fbox{\Large\moddate}}              %
%%%%%%%%%%%%%%%%%%%%%%%%%%%%%%%%%%%%%%%%%%%%%%%%%%%%%%%%%%%%%%%%%%%

\renewcommand{\strut}{\rule{0pt}{.85em}}

\begin{document}

\maketitle

\section{Introduction}

A \emph{front end} performs a \tlaplus\ to \tlaplus\ translation of an
obligation for some class of backend provers.  Its main function is to
hide from a backend prover operators that the prover can't handle, or
would handle incorrectly.  A front end is a device for describing what
TLAPS does.  It is not necessarily implemented as a separate component
of the proof manager.

We want the translation to be sound, meaning that if the translated
obligation is provable then the original obligation is valid.  One way
to do that is to ensure that the translated obligation is equivalent
to the original.  The translation may introduce new uninterpreted
symbols.  In that case, we show soundness by showing that the
translation is \emph{definably equivalent} to the original, meaning
that there exist definitions of the new symbols that make it
equivalent to the original.

For now, we seem to need the following front ends (the names are not
fixed).
\begin{description}
\item[Action] This is used for ordinary action reasoning.  It hides
temporal operators and a class of operators called non-Leibniz
operators, described below.

\item[PTL] This is used for PTL (propositional temporal logic)
reasoning, the resulting obligations sent to a PTL backend prover.  It
hides all operators except those of PTL.

\item[TLA] This is used for temporal predicate logic (quantification
over constants [rigid variables]) and application of temporal proof
rules, the resulting obligations sent to a backend for temporal
reasoning that has yet to be implemented.

\item[Enabled] This is a front end used in conjunction with the Action
front end for reasoning about the \tlaplus\ \textsc{enabled} operator.
Instead of hiding \textsc{enabled}, this front end eliminates it from
an obligation by expandingits definition.

\end{description}
These front ends are described in Section~\ref{sec:front-ends}.
Central to front ends are the concepts of 
\emph{coalescing} and \emph{Leibnizing}, which we discuss after
introducing a bit of notation.

TLAPS processes each proof obligation independently.  We therefore
assume a single original obligation that is being transformed to its
translated version.



%

\section{Preliminaries}

We assume some notion of \emph{syntactic equivalence} of expressions
and operator symbols that satisfies the following conditions:
\begin{itemize}

\item Any operator symbol or expression is syntactically equivalent
to itself.

\item 
$F(\oneto{d}{n}) = G(\oneto{e}{n})$ if
$F$, \oneto{d}{n} are syntactically equivalent to $G$, \oneto{e}{n},
respectively.
\end{itemize}

For any expressions or operator symbols \oneto{e}{n}, we let
\B{\oneto{e}{n}} be an identifier not appearing in the obligation such
that \B{\oneto{d}{m}} and \B{\oneto{e}{n}} are the same identifier iff
$m=n$ and $d_{i}$ is syntactically equivalent to $e_{i}$, for all $i$ in
$1\dd m$.

\section{Coalescing}

In its simplest form, coalescing is replacing an expression $exp$ by
\B{exp}.  We expect to show that the new obligation produced
in this way is definably equivalent to the original one under the
definition
 \[ \B{exp} == exp\]
However, it need not be in the presence of bound identifiers.  For
example, suppose we define $P(a)=a$, so \mbox{$\A b : P(b)=b$} is a valid
formula.  This way of coalescing the expression $P(b)$ in that formula
produces the formula \mbox{$\A b: \B{P(b)}=b$}, which is not valid.
(Remember that \B{P(b)} is just a symbol, bearing no relation to the
symbol $b$, so from this formula a prover can deduce $\B{P(b)}=1$
and $\B{P(b)}=2$, proving $1=2$.)

In general, suppose $exp$ is an expression in which \oneto{id}{m} are
all the identifiers occurring in $exp$ that are bound in the context
in which $exp$ appears.  We then coalesce $exp$ as $\B{exp,
\oneto{id}{m}}(\oneto{id}{m})$.  We obtain definable equivalence with
the definition
  \[ \B{exp, \oneto{id}{m}}(\oneto{id}{m}) == exp\]
Note that we can use any ordering of the identifiers $id_{i}$.
Changing the ordering changes both the identifier \B{exp,
\oneto{id}{m}} and its definition.

To make the translated obligation as easy as possible to prove, we
would like to use the same new symbol to coalesce as many different
expression occurrences as possible.  We can try to do this by
coalescing the expression $exp$ as if the identifiers $id_{i}$ had
been replaced by some standard symbols.  However, this is unlikely to
help in practice.  If two expressions can be coalesced with the same
new operator, it's best to count on the user writing them in a
syntactically equivalent form rather than hoping to make them
syntactically equivalent by some standard transformation.  The only
thing a front-end transformation should do is to make the ordering of
the $id_{i}$ in the coalescing depend only on the identifiers
themselves---e.g., by putting them in alphabetical order.

\section{Leibnizing}
\begin{flushright}
\emph{You can verb anything.}\\
Ron Ziegler (quoted by Brian Reid)
\end{flushright}
%
\noindent The \emph{Leibniz principle} asserts that for any
expressions $d$, $e$, and $F$, the formula $(d=e)=>(F =
F^{e}_{\!\!\!d})$ is valid, where $F^{e}_{\!\!\!d}$ is the expression
obtained from $F$ by replacing all instances of $d$ by $e$.  The
Leibniz principle does not hold for a modal logic like TLA\@.  For
example, if $x$ and $y$ are variables, then the Leibniz principle
implies $(x=y)=>(x'=y')$, which is not in general valid.

The Leibniz principle can be violated by an expression $exp$ only if
the expression obtained from $exp$ by (recursively) expanding all user
definitions contains a non-constant operator of \tlaplus---that is,
one of the action operators or temporal operators listed in Tables~3
and~4 on page 269 of \emph{Specifying Systems}.  (This is well-defined
because, in \tlaplus, a recursive definition is just an abbreviation
for a more complicated non-recursive one.)

Most back-end provers assume the Leibniz principle, meaning that they
will freely substitute equals for equals anywhere they can.  They are
prevented from doing that by a front-end transformation.  For example,
the Action front-end coalesces $x'$ and $y'$ to \B{x'} and \B{y'},
so $(x=y)=>(x'=y')$ becomes $(x=y)=>(\B{x'}=\B{y'})$, which cannot
be proved. 

Unexpanded user-defined operators pose a problem.  The user
could define 
  $P(a) \deq a'$
and write $(x=y)=>(P(x)=P(y))$, an expression that contains no
explicit primes.  We say that an operator $F$ is Leibniz if
 $(d=e)=>(P(d)=P(e))$
is valid for any expressions $d$ and $e$.  (The definition for
operators with multiple arguments is obvious.)  \emph{Leibnizing} is
the procedure of replacing an occurrence $P(x)$ of a non-Leibniz
operator $P$ by $\B{\ldots}(x)$, which is equivalent to $P(x)$ when
\B{\ldots} is defined to be some Leibniz operator.  We explain below
what the ``\ldots'' is.



\subsection{Some Definitions}

We recursively define what it means for the $i$\tth\ argument position
of an operator $Op$ to be Leibniz as follows:
\begin{itemize}
\item All argument positions of the \tlaplus\ built-in constant
operators are Leibniz and all argument positions of its non-constant
operators are non-Leibniz.

\item For an operator $Op$ defined by $Op(\oneto{a}{n}) \deq def$, the
$i$\tth\ argument position of $Op$ is Leibniz iff $a_{i}$ does not
appear in $def$ within a non-Leibniz argument position of an operator.
\end{itemize}
An operator is Leibniz, in the sense defined above, if all of its
argument positions are Leibniz.  The converse is true in most
practical examples.  Therfore, by a Leibniz operator, we now mean one
all of whose argument positions are Leibniz.

Higher-order operators complicate matters.  Consider the operator
 \[ G(F(\_), a) == F(a)
 \]
Both argument positions of $G$ are Leibniz according to our definition
above, but
 \[ (x=y) \;=> \;(G(\,'\,,\,x) = G(\,'\,,\, y)) \]
is not in general valid.  As this shows, whether or not the Leibniz
principle applies to an ordinary (non-operator) argument can depend on
what operators occur in its operator arguments.  There is no problem
if we only 

Let $Op$ be defined as above.  We say that (ordinary) argument
position $i$ of $Op$ occurs in position $j$ of its $k$\tth\ argument
iff $a_{i}$ appears in $def$ in the $j$\tth\ argument of the
(operator) argument $a_{k}$.  We say that the $i$\tth\ argument
of the expression $Op(\oneto{e}{n})$ is \emph{non}-Leibniz
iff either:
\begin{itemize}
\item The $i$\tth\ argument of $Op$
is an operator argument and either the $i$\tth\ argument
position of $Op$ is non-Leibniz or $e_{i}$ is a non-Leibniz operator.

\item The $i$\tth\ argument of $Op$ is an ordinary argument and either
the $i$\tth\ argument position of $Op$ is non-Leibniz, or for some $j$
and $k$, argument position $i$ of $Op$ occurs in position $j$ of its
$k$\tth\ argument and the $j$\tth\ argument position of $e_{k}$ is
non-Leibniz.
\end{itemize}


\subsection{Leibniz for Constants}

Although \tlaplus\ doesn't satisfy the Leibniz principle, it satisfies
the Leibniz principle for constants.  More precisely:
\begin{itemize}
\item[] The formula $(d=e) => (F(d) = F(e))$ is valid for any constant
expressions $d$ and $e$ and any operator $F$ definable in \tlaplus.
\end{itemize}


\subsection{How to Leibniz}


We replace the application $Op(\oneto{e}{n})$ of a non-Leibniz operator
$Op$ by 
 \[ \B{Op,\oneto{\epsilon}{n}}(\oneto{e}{n})
 \]
where 
\begin{tabbing}
\s{2} $\epsilon_{i}$ \  \= = \ $e_{i}$ \= \ if the $i$\tth\ expression of
                  $Op(\oneto{e}{n})$ is non-Leibniz \\
     \> \> and $e_{i}$ is not a constant expression\\
     \> = $\tau$ \ \> \ otherwise
\end{tabbing}
where $\tau$ is a fixed new symbol (the same symbol for all $i$ and
all Leibnizings).

\begin{sloppypar}
To show definable equivalence, we have to define a Leibniz operator
\B{Op,\oneto{\epsilon}{n}} for which
$\B{Op,\oneto{\epsilon}{n}}(\oneto{e}{n})$ is equivalent to
$Op(\oneto{e}{n})$.  In the definition, how each argument $e_{i}$
is handled depends on which of the following kinds of argument it is:
\end{sloppypar}
\begin{enumerate}
\item A Leibniz argument.

\item A non-Leibniz argument that is a constant expression.

\item A non-Leibniz argument that is either an operator or a non-constant
      expression.
\end{enumerate}
We illustrate this with the expression $F(x\!+\!1, 22, x\!+\!3)$ where
$x$ is a variable and 
 \[ F(a, b, c) == a + (b + c)'\]
so its first argument position is Leibniz and its last two are
non-Leibniz.  We then define
  \[  \B{F,\tau, \tau, x\!+\!3}(a, b, c) == 
      \CHOOSE z : \E bb : \begin{conj}
                          b = bb \\ z = F(a, bb,\, x\!+\!3)
                          \end{conj}
  \]
It is easy to check that \B{F,\tau, \tau, x\!+\!3} is Leibniz in all its
arguments, because none of the formal parameters $a$, $b$, or $c$ occur
in a non-Leibniz argument position.  That
 \[ \B{F,\tau, \tau, x\!+\!3}(x\!+\!1, 22, x\!+\!3) = F(x\!+\!1, 22, x\!+\!3)
 \]
follows easily from the fact that any operator satisfies the Leibniz
principle for constants, and the bound identifier $bb$ in the
definition is a constant.  Finally, we must show that the Leibnizing
of any other expression that yields the same operator symbol
\B{F,\tau, \tau, x\!+\!3} produces the same definition.  This is
obvious.
Observe that this Leibnizing allows a prover to deduce
  \[ (x=y) => (F(x\!+\!1, 22, x\!+\!3) = F(y\!+\!1, 20\!+\!2, x\!+\!3))
 \]
Note: since \textsc{choose} can be applied only to action-level
formulas, this definition wouldn't work if $F(x\!+\!1, 22, x\!+\!3)$
were a temporal-level expression.  However, the syntax of \tlaplus\
implies that temporal-level expressions are formulas, so in that
case we could change the definition to:
  \[  \B{F,\tau, \tau, x\!+\!3}(a, b, c) == 
      \E bb : \begin{conj}
                          b = bb \\ F(a, bb,\, x\!+\!3)
                          \end{conj}
  \]

Leibnizing replaces an operator application by a definably equal one.
However, that doesn't necessarily produce a definably equal obligation
if those operator applications occur in a non-Leibniz argument.  
Therefore, Leibnizing needs to be done in a top-down fashion---Leibnizing
an operator application $Op(\oneto{e}{n})$
before Leibnizing operator applications in the arguments $e_{i}$.


\section{The Front Ends} \label{sec:front-ends}

\begin{sloppypar}
One operation that a front end may perform is distributing primes over
operators.  Currently this is done by replacing $Op(\oneto{e}{n})'$
with $Op(\oneto{e'}{n})$ for any constant operator $Op$.  For any
non-constant Leibniz operator $Op$ with no operator arguments, we can
also transform $Op(\oneto{e}{n})'$ to $\B{Op'}(\oneto{e'}{n})$.  We
can define
 $\B{Op'}(\oneto{a}{n})$
to be the expression obtained by fully expanding all definitions in
 $Op(\oneto{a}{n})'$, distributing primes over constant
\tlaplus\ operators, and replacing each $a_{i}'$ by $a_{i}$.
This defines $\B{Op'}(\oneto{e'}{n})$ to equal
  $Op(\oneto{e}{n})'$
because since $Op$ is Leibniz, each $a_{i}$ occurs only as arguments
of constant operators in the definition.
(It's not clear how useful this enhancement will be in practice.)
\end{sloppypar}

A front end transforms the assumptions and the goal of the obligation.
It first expands usable user-defined operations and perhaps some
built-in \tlaplus\ operators.  It may then distribute primes over
operators.  It then may coalesce expressions and Leibniz operator
applications.  An operator application must be Leibnized before any
subexpression of its arguments can be coalesced or Leibnized.  (It is
safe to coalesce or Leibniz arguments of a Leibnized operator
application because the newly introduced operator is Leibniz.)



\subsection{The Action Front End}

This is the default front end, used for action reasoning.  It expands
the definitions of the operators \textsc{unchanged}, $[A]_{v}$ and
$<<A>>_{v}$; and it distributes primes over operators.  It then
coalesces all expressions whose principal operator is a non-constant
\tlaplus\ operator and Leibnizes non-Leibniz operator applications.

\subsection{The PTL Front End}

This first expands the definitions of $[A]_{v}$, $<<A>>_{v}$, $~>$,
WF, and SF. It then coalesces every expression whose principal
operator is not $[]$, $<>$, $\,'\,$\,.  or an operator of
propositional logic ($/\ $, $~$, $=>$, etc.).  The \,$'$\, operator is
given to the backend provers as the LTL (Linear-time Temporal Logic)
\emph{next} operator.  (After coalescing, all remaining primes are
priming formulas.)



\subsection{The TLA Front End}

This is for a backend that can handle temporal formulas
including quantification over constants (rigid variables).  A minimum
requirement is that it must be able to apply the tautology:
 \[  (\E i : <>P(i)) \,\equiv\, <> (\E i : P(i))
 \]
The backend is under development, so we do not yet know exactly what its
front end will do.


\subsection{The Enabled Front End}

This frontend eliminates \textsc{enabled} expressions by expanding the
definition of \textsc{enabled}.  It is used in conjunction with the
Action front end.  Let $\oneto{x}{m}$ be the declared
\textsc{variables} of the module.  The basic idea is to replace
$\textsc{enabled} A$ with
 \[ \E \B{x_{1}'},\, \ldots,\, \B{x_{m}'} : \overline{A}
 \]
where $\overline{A}$ is obtained from $A$ approximately by replacing
each instance of each $x_{i}'$ with \B{x_{i}'}.  There are two reasons
for the ``approximately'': nested \textsc{enabled} expressions and
user-defined operators.  

Handling nested \textsc{enabled}s is straightforward.  They are
eliminated in a bottom-up fashion, renaming the bound identifiers
\B{x_{1}'} as needed to produce a valid \tlaplus\ expression.

\begin{sloppypar}
User-defined operators can ``hide'' instances of primed variables
$x_{i}'$ in their definitions, and those hidden instances must be made
visible as instances of \B{x_{i}'} when expanding the definition of
$\textsc{enabled}$.  For example, if $B$ is defined to equal 
 \mbox{$x' \in S$}, 
then $\ENABLED B$ must be replaced by 
 \mbox{$\E \B{x'} : \B{B}(\B{x'})$}.
\end{sloppypar}

It is always sound to replace any operator application 
$Op(\oneto{e}{n})$ with 
   $\B{Op}(\oneto{e}{n}, \B{x_{1}'},\,\ldots,\, \B{x_{m}'})$.
However, we would like to add as arguments only those
\B{x'_{i}} for which $x_{i}'$ actually occurs in the expansion
of $Op(\oneto{e}{n})$.  For example, this would allow a prover
to deduce 
  \[ (\ENABLED A /\ B) \; \equiv (\ENABLED A) \, /\ \, (\ENABLED B)
 \]
if $A$ and $B$ have no primed variables in common.

When expanding an \textsc{enabled} expression, the front-end will
replace any occurence $Op(\oneto{e}{n})$ of a Leibniz operator $Op$
that does not occur within a primed expression by
  \[ \B{Op}(\oneto{e}{n}, \B{x_{i_{1}}'},\,\ldots,\, \B{x_{i_{k}}'})
  \]
where the $x'_{i_{j}}$ are all the primed variables that actually
occur in the expansion of the definition of $Op$.

It is possible to do this for the other cases as well: $Op$
non-Leibniz or the operator application occurring within a primed
expression.  However, this is probably not worth doing because those
cases should rarely occur in practice.



\end{document}

%%%%%%%%%%%%%%%%%%%%%

\subsection{Notation}

We introduce some rather unusual notation that will take some getting
used to.  However, it allows us to do things more generally and we
hope will smooth the path to implementing the ideas.

For any finite set $S$, let $|S|$ be the cardinality of $S$.  If
$S$ is the set $\{i_{1}, \ldots , i_{j}\}$ of integers with
 $i_{1} < \ldots < i_{j}$,
define $v_{S}$ to be an abbreviation for 
  $v_{i_{1}}, \ldots, v_{i_{j}}$.  
%
% For $j$ in $1\ldots |S|$, we define \Sj{S}{j} to equal $i_{j}$.
Thus, 
\begin{itemize}
\item $\E x_{1\!\ldots 3} : P(x_{1\!\ldots 3})$ \ is an abbreviation for \
$\E x_{1}, x_{2}, x_{3}: P(x_{1}, x_{2}, x_{3})$.

\item  $<<x_{1\!\dd n},\,y_{1\!\dd m}>>$ \ means \
$<<x_{1}, \ldots, x_{n}, y_{1}, \ldots, y_{m}>>$.  


\item $<<x_{S}>> = <<y_{S}>>$ \ is equivalent to 
the assertion that $x_{i}=y_{i}$ for all $i$ in $S$.

% \item $\Sj{1\dd n}{j} \ = \ j$, for all $j \in 1\dd n$.

% \item \Sj{\{1, 3, 5\}}{2} \ equals \ 3.

% \item $<<x_{S}>>$ \ equals \ $[j \in 1\dd |S| |-> x_{\Sj{S}{j}}]$.
\end{itemize}
%
We let these expressions have the obvious meanings when $S$ is the empty set.
For example,
\begin{itemize}
\item $\E x_{\{\}} : P(x_{\{\}})$ \ means \ $P$.

\item $<<x_{1\!\dd n},\,y_{1\!\dd m}>>$ \ means \ $<< >>$ \ if $m=n=0$.
\end{itemize}
%
If
$S$ is the set $\{i_{1}, \ldots , i_{j}\}$ of integers with
 $i_{1} < \ldots < i_{j}$ and $\pi$ is a permutation of $1\dd j$, we
define $v_{\pi(S)}$ to equal $v_{i_{\pi(1)}}, \ldots, v_{i_{\pi(j)}}$.

Let \,$e \WITH id <- d$\, be the expression obtained from the expression $e$
by substituting the expression $d$ for the identifier $id$.  We define
\ $e \WITH id_{1\!\dd n} <- d_{1\!\dd n}$ \ in the obvious way to equal
 \[ (\ldots(e \WITH id_{1} <- d_{1}) \WITH \ldots) \WITH id_{n} <- d_{n}
 \]
More generally, this defines $e \WITH id_{S} <- d_{T}$ for any index
sets $S$ and $T$ with $|S| = |T|$.

For any expression $e$, define \X{e} to be the expression obtained
from $e$ by recursively using definitions to expand all occurrences of
user-defined operators.  The following is a consequence of the meaning
of definitions:
\begin{observation}\label{obs:def} \rm
For any user-defined operator $Op$ of $n$ arguments (some of which may
be operator arguments) and any expression $e$ and expressions or
operators $e_{1\!\dd n}$\,:
\begin{enumerate}
\item $e$ equals $\X{e}$.
 
\item If $Op$ is defined by
 \,$Op(\alpha_{1\!\dd n}) \deq d$\,,
then
 \[ \X{Op(e_{1\!\dd n})} \;=\; \X{d} \WITH \alpha_{1\!\dd n} <- e_{1\!\dd n}\]
\end{enumerate}
\end{observation}
%
Let an \emph{ordinary} operator be one all of whose arguments are
expressions; and let a \emph{higher-order} operator be one having at
least one operator argument.  In Observation~\ref{obs:def}, if
$\alpha_{i}$ is an operator argument then the corresponding $e_{i}$ is
either an operator name or a \textsc{lambda} expression.

\subsection{Semantics}

The meaning of a \tlaplus\ expression is a mapping from behaviors to
values.  A behavior is a sequence of states, where a state is a
mapping from variables to values.  Since \tlaplus\ is based on ZFC set
theory, a value is a set.  

For any behavior $\sigma$, define $\sigma_{i}$ to be the $i$\tth\
state of $\sigma$, so $\sigma$ equals $\sigma_{1}, \sigma_{2},\ldots$\,.  Let
$\sigma^{+n}$ be the behavior $\sigma_{n+1}, \sigma_{n+2}, \ldots$.

Let \M{e} be the meaning of the expression
$e$, so $\M{e}(\sigma)$ is a value, for any behavior $\sigma$.
The meaning of an ordinary \tlaplus\ operator is a mapping from
mappings of behaviors to values to a mapping from behaviors to values.
For example, \M{\Box} is defined by 
 \[ \M{\Box(F)}(\sigma) == \A n \in Nat : \M{F}(\sigma^{+n})
 \]


