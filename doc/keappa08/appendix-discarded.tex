%%% -*- mode: LaTeX; TeX-master: "main.tex"; -*-

\ifx\master\undefined
\documentclass[a4paper]{easychair}
\usepackage{submission}
\begin{document}
{\let\master\relax%%% -*- mode: LaTeX; TeX-master: "main.tex"; -*-

\ifx\master\undefined
\documentclass[a4paper]{easychair}
\usepackage{submission}
\begin{document}
\fi
%%%% PLEASE DO NOT EDIT ABOVE THIS LINE

\title{A \tlaplus Proof System}

\titlerunning{A \tlaplus Proof System}

% \volumeinfo
% 	{P. Rudnicki, G. Sutcliffe} % editors
% 	{2}                         % number of editors
% 	{KEAPPA 2008}               % event
% 	{1}                         % volume
% 	{1}                         % issue
% 	{1}                         % starting page number


%  Alphabetically by surname
\author{
  Kaustuv Chaudhuri \\
  INRIA \\
  \and
  Damien Doligez \\
  INRIA \\
  \and
  Leslie Lamport \\
  Microsoft Research \\
  \and
  Stephan Merz \\
  INRIA \& Loria
}

\authorrunning{Chaudhuri, Doligez, Lamport, and Merz}

\maketitle

%%%% PLEASE DO NOT EDIT BELOW THIS LINE
\ifx\master\undefined
{\let\master\relax %%% -*- mode: LaTeX; TeX-master: "main.tex"; -*-

\ifx\master\undefined
\documentclass[a4paper]{easychair}
\usepackage{submission}
\begin{document}
{\let\master\relax %%% -*- mode: LaTeX; TeX-master: "main.tex"; -*-

\ifx\master\undefined
\documentclass[a4paper]{easychair}
\usepackage{submission}
\begin{document}
\fi
%%%% PLEASE DO NOT EDIT ABOVE THIS LINE

\title{A \tlaplus Proof System}

\titlerunning{A \tlaplus Proof System}

% \volumeinfo
% 	{P. Rudnicki, G. Sutcliffe} % editors
% 	{2}                         % number of editors
% 	{KEAPPA 2008}               % event
% 	{1}                         % volume
% 	{1}                         % issue
% 	{1}                         % starting page number


%  Alphabetically by surname
\author{
  Kaustuv Chaudhuri \\
  INRIA \\
  \and
  Damien Doligez \\
  INRIA \\
  \and
  Leslie Lamport \\
  Microsoft Research \\
  \and
  Stephan Merz \\
  INRIA \& Loria
}

\authorrunning{Chaudhuri, Doligez, Lamport, and Merz}

\maketitle

%%%% PLEASE DO NOT EDIT BELOW THIS LINE
\ifx\master\undefined
{\let\master\relax \input{rearmatter}}
\end{document}
\fi

% LocalWords:  tex Rudnicki Sutcliffe KEAPPA Kaustuv Chaudhuri INRIA Doligez
% LocalWords:  Merz Loria
}
\fi
%%%% PLEASE DO NOT EDIT ABOVE THIS LINE

\bibliographystyle{plain}
\bibliography{submission}

%%%% PLEASE DO NOT EDIT BELOW THIS LINE
\ifx\master\undefined
\end{document}
\fi

% LocalWords:  tex Paxos
}
\end{document}
\fi

% LocalWords:  tex Rudnicki Sutcliffe KEAPPA Kaustuv Chaudhuri INRIA Doligez
% LocalWords:  Merz Loria
}
\fi
%%%% PLEASE DO NOT EDIT ABOVE THIS LINE

\clearpage
\appendix

\section{The \PM Specification}
\label{apx}

We shall now give a somewhat formal declarative specification of the
\PM and prove the key correctness theorem~\ref{thm:meaning} (a
generalization of which appears as Thm.~\ref{thm:correctness}).

\subsection{Context Management}
\label{apx:context}

In this section we shall provide the details of all context management
operations performed by the \PM. First, let us recall the main
syntactic categories.

\begin{defn}[Meta-syntax] \label{defn:syntax}
  %
  \emph{Assertions}, \emph{contexts}, \emph{assumptions} and
  \emph{definables} have the following syntax, where "e" ranges over
  \tlatwo expressions, "x" over \tlatwo identifiers, "\vec x" over
  lists of \tlatwo identifiers, and "o" over \tlatwo operator names
  %
  \begin{quote}
    \begin{tabbing}
      (Assertions) \SP \= "\phi" \SP \LSP \= "::=" \ \= "\obl{\G ||- e}" \\
      (Contexts) \> "\G, \D, \dotsc" \> "::=" \> "\cdot OR \G, h" \\
      (Assumptions) \> "h" \> "::=" \> "\NEW x OR o \DEF \delta OR \phi OR \hide{o \DEF \delta} OR \hide{\phi}" \\
      (Definables) \> "\delta" \> "::=" \> "\phi OR \LAMBDA\ \vec x : e"
    \end{tabbing}
  \end{quote}
  %
  The expression after "||-" in an assertion is called its
  \emph{conclusion}.  Hidden assumptions are written inside square
  brackets "\hide{\ }". As usual, "\obl{\cdot ||- e}" is written
  simply as "e", and context adjunction is written with commas (that
  is, "\G, \D" stands for "\G, h_1, ..., h_n" if "\D" is "\cdot, h_1,
  ..., h_n"). The assumptions "\NEW x", "o \DEF \delta" and "\hide{o
    \DEF \delta}" bind the identifiers "x" and "o" respectively to
  their right in their contexts and conclusions. We say that "x \in
  \G" if "x" is bound in "\G", and "x \notin \G" if "x" is not bound
  in "\G".
\end{defn}

As mentioned in section~\ref{sec:backend.pm}, hidden facts and
definitions are filtered out of assertions before an assertion is sent
to the back-end. Hidden assertions are simply deleted, while hidden
operator definitions "\hide{o \DEF \delta}" are treated as if they
were replaced with declarations "\NEW o". The precise embedding into
Isabelle/\tlaplus propositions is as follows.

\begin{defn} \label{defn:isabelle-embedding}
  %
  The Isabelle/\tlaplus \emph{embedding} "\isa{-}" of assertions,
  contexts and definables is as follows:
  %
  \begin{align*}
    && \isa{"\G ||- e"} &= "\isa{\G} e" \\
    \isa{"\G, \NEW x"} &= "\isa{\G} \And x." & \LSP
    \isa{"\G, o \DEF \delta"} &= \isa{\G} "o \equiv \isa{\delta} ==>" & \LSP
    \isa{"\G, \phi"} &= "\isa{\G} \isa{\phi} ==>" \\
    \isa{\cdot} &= &
    \isa{"\G, \hide{o \DEF \delta}"} &= \isa{\G} \And o. & \LSP
    \isa{"\G, \hide{\phi}"} &= "\isa{\G}" \\
    &&\isa{"\LAMBDA\ \vec x : e"} &= "\lambda \vec x.\ e"
  \end{align*}
\end{defn}

%\noindent
For example, "\isa{\NEW P, \obl{\NEW x ||- P(x)} ||- \forall x : P(x)}
= \And P.\ \left(\And x.\ P(x)\right) ==> \forall x : P(x)". Note that
the embedding of ordinary \tlatwo expressions is the identity because
Isabelle/\tlaplus contains \tlatwo expressions as part of its object
syntax. In practice, some aspects of \tlatwo expressions, such as the
indentation-sensitive conjunction and disjunction lists, are sent by
the \PM to Isabelle using an indentation-insensitive
encoding. Moreover, as already mentioned, Isabelle/\tlaplus accepts
only constant level expressions.

\begin{defn}[Truth] \label{defn:true} \mbox{}
  %
  The assertion "\obl{\G ||- e}" is said to be \emph{true} if "\isa{\G
    ||- e}" is a well-typed true proposition of Isabelle/\tlaplus.
\end{defn}

We state, omitting the trivial proofs, a number of facts about
assertions in Isabelle/\tlaplus.

\begin{fac}[Definition] \label{thm:definition}
  % 
  If "\obl{\G, \NEW o, \D ||- e}" is true, then "\obl{\G, o \DEF
    \delta, \D ||- e}" is true.
\end{fac}

\begin{fac}[Instantiation] \label{thm:inst}
  %
  If "\obl{\G, \NEW o, \D ||- e}" is true, then "\obl{\G, \D[o :=
    \delta] ||- e[o := \delta]}" is true.
\end{fac}

\begin{fac}[Weakening] \label{thm:weaken}
  %
  If "\obl{\G, \D ||- e}" is true, then "\obl{\G, h, \D ||- e}" is
  true.
\end{fac}

\begin{fac}[Strengthening] \label{thm:delete}
  %
  If "\obl{\G, \NEW o, \D ||- e}" or "\obl{\G, o \DEF \delta, \D ||-
    e}" or "\obl{\G, \hide{o \DEF \delta}, \D ||- e}" is true and "o"
  is not free in "\obl{\D ||- e}", then "\obl{\G, \D ||- e}" is true.
\end{fac}

\begin{fac}[Cut] \label{thm:cut}
  %
  If "\obl{\G, \D ||- e}" and "\obl{\G, (\D ||- e) ||- f}" are true,
  then "\obl{\G ||- f}" is true.
\end{fac}

The \USE/\HIDE \DEFS steps can expose or hide operator definitions in
the context of their assertion, which we define with the help of the
\kwd{using} and \kwd{hiding} meta-syntactic keywords respectively.

\begin{defn} \label{defn:use/hide}
  %
  If "\G" is a context and "\P" a list of operator names, then:
  \begin{ecom}
  \item \emph{"\G" with "\P" marked usable}, written "\G \USING \P",
    is defined by the following equations.
    \begin{align*}
      \G \USING\ \cdot &= \G \\
      \G \USING \P, o &= \G \USING \P \LSP \text{if } o \notin \G \\
      \G, \NEW x, \D \USING \P, x &= \G, \NEW x, \D \USING \P \\
      \G, \hide{o \DEF \delta}, \D \USING \P, o &= \G, o \DEF \delta, \D \USING \P \\
      \G, o \DEF \delta, \D \USING \P, o &= \G, o \DEF \delta, \D \USING \P
    \end{align*}
  \item \emph{"\G" with "\P" marked hidden}, written "\G \HIDING \P",
    is defined by the following equations:
    \begin{align*}
      \G \HIDING\ \cdot &= \G \\
      \G \HIDING \P, o &= \G \HIDING \P \LSP \text{if } o \notin \G \\
      \G, \NEW x, \D \HIDING \P, x &= \G, \NEW x, \D \HIDING \P \\
      \G, \hide{o \DEF \delta}, \D \HIDING \P, o &= \G, \hide{o \DEF \delta}, \D \HIDING \P \\
      \G, o \DEF \delta, \D \HIDING \P, o &= \G, \hide{o \DEF \delta}, \D \HIDING \P
    \end{align*}
  \end{ecom}
\end{defn}

% \ednote{SM}{I have trouble making sense of (indeed, even parsing) the
%   lines containing ``$= \Omega$'' (also, $\Omega$ has not been
%   introduced as one of the syntactic categories).\\
% DD: I removed them as I am quite sure they are leftovers from a
% change of notation.}
%
% oops,and thanks. -- KC

\noindent
The effect of \USE/\HIDE \DEFS is characterized by means of the
following lemma.

\begin{lem}[Using/Hiding Definitions] \label{thm:use/hide-defs} \mbox{}
  %
  \begin{ecom}
  \item If "\obl{\G ||- e}" is true then
    "\obl{\G \USING \P ||- e}" is true.
  \item If "\obl{\G \HIDING \P ||- e}" is true then
    "\obl{\G ||- e}" is true.
  \end{ecom}
\end{lem}

% \ednote{SM}{Simplified the assertions of the preceding lemma, in light
%   of the following edits. The original versions are still there.}
%
% Thanks. I agree with the change. -- KC

\begin{proof}
  In either case, an immediate consequence of
  fact~\ref{thm:definition}.
\end{proof}

\def\refl#1{\bigl\|\,#1\,\bigr\|}

A sequence of binders "\vec \beta" in the \tlatwo expressions "\forall
\vec \beta : e" or "\exists \vec \beta : e" can be reflected as
assumptions.

\begin{defn}[Binding Reflection] \label{defn:binding-reflection}
  %
  If "\vec \beta" is a list of binders with each element of the form
  "x" or "x \in e", then the \emph{reflection} of "\vec \beta" as
  assumptions, written "\refl{\vec \beta}", is given inductively as
  follows.
  \begin{align*}
    \refl{\cdot} &= \cdot &
    \refl{\vec \beta, x} &= \refl{\vec \beta}, \NEW x &
    \refl{\vec \beta, x \in e} &= \refl{\vec \beta}, \NEW x, x \in e
  \end{align*}
\end{defn}

\subsection{Computing Assertions}
\label{apx:proof-transformation}

As mentioned in section~\ref{sec:obligations}, the two claim forms
"\OBVIOUS : \obl{\G ||- e}" and "\OMITTED : \obl{\G ||- e}" are said
to be primitive. The \PM performs two operations on \tlatwo proofs and
assertions, \textit{checking} and \textit{transformation}, in order to
determine the set of primitive claims for a claim. Each procedure is
declaritively specified below; note that the specifications are
mutually recursive.

\begin{defn}[Checking] \label{defn:checking}
  % 
  The \tlatwo proof "\pi" \emph{checks} the assertion "\obl{\G ||- e}"
  if there is a derivation of "\pi : \obl{\G ||- e}" using the
  following inference rules whose leaves are primitive claims.
  %
  \begin{gather*}
    \I[\BY]{"\BY\ \Phi\ \DEFS\ \P : (\G
      ||- e)"}
      {"\s{0}.\ \USE\ \Phi\ \DEFS\ \P" : "(\G ||- e) --> (\D ||- f)"
       &
       "\OBVIOUS : (\D ||- f)"}
    \SP
    \I[\QED]{"\sigma.\ \QED\ \PROOF\ \pi : (\G ||- e)"}
      {"\pi : (\G ||- e)"}
    \\[1ex]
    \I[non-\QED ("n > 1")]{"\sigma_1.\,\tau_1\ \ \sigma_2.\,\tau_2\ \ \dotsb\ \ \sigma_n.\,\tau_n : (\G ||- e)"}
      {"\sigma_1.\,\tau_1 : (\G ||- e) --> (\D ||- f)"
       &
       "\sigma_2.\,\tau_2\ \dotsb\ \sigma_n.\,\tau_n : (\D ||- f)"}
  \end{gather*}  
\end{defn}

\ednote{SM}{Why is there no step number in the conclusion of the \BY{}
  rule? I presume that the use of $\s{0}$ in the premise is
  arbitrary?}

\ednote{KC}{I don't understand your first question. A \BY is not a
  proof step but a proof. The terminology I am using, which I believe
  is also Leslie's terminology, is that a proof "\pi" of level "n",
  written "\pi(n)", is defined by the following syntax (roughly):
  \begin{quote} \itshape
    \begin{tabbing}
      (proofs) \hspace{4em}  \= "\pi(n)" \quad \= "::=" \quad \= "\OBVIOUS OR \OMITTED OR \BY\ \Phi\ \DEFS\ \P OR \Pi(n)" \\[1ex]
      (non-leaf proofs) \> "\Pi(n)" \> "::=" \> "\sigma(n).\ \QED\ \PROOF\ \pi(m)" \` ("m > n") \\
                   \>          \> "\ \ |"  \> "\sigma(n).\ \tau(n)\quad \Pi(n)" \\[1ex]
      (proof steps) \> "\tau(n)" \> "::=" \> "\USE\ \Phi\ \DEFS\ \P OR \HIDE\ \Phi\ \DEFS\ \P OR \DEFINE\ o \DEF e" \\
                                               \` (proof context mutation) \\[1ex]
                    \>         \> "\ \ |" \> "\HAVE\ e OR \TAKE\ \vec \beta OR \WITNESS\ \vec e" \` (steps without subproofs) \\[1ex]
                    \>         \> "\ \ |" \> "\obl{\G ||- e}\ \PROOF\ \pi(m) OR \SUFFICES\ \obl{\G ||- e}\ \PROOF\ \pi(m)" \\
                    \>         \> "\ \ |" \> "\PICK\ \vec \beta : e\ \PROOF\ \pi(m)" \` (steps with subproofs, "m > n") \\[1ex]
      (begin-step tokens) \> "\sigma(n)" \> "::=" \> "\s n OR \s n l"
    \end{tabbing}
  \end{quote}
  If it helps to make the text clearer, we can add a definition such
  as the above earlier on, especially now that the table of \tlatwo
  proof constructs has been removed from
  section~\ref{sec:proof-language}.

  \medskip
  On the second question, yes, the 0 is arbitrary.}


\begin{defn}[Transformation] \label{defn:transform}
  %
  The \tlatwo proof step "\sigma.\,\tau" \emph{transforms} the
  assertion "\obl{\G ||- e}" into "\obl{\D ||- f}" if there is a
  derivation of "\sigma.\ \tau : \obl{\G ||- e} --> \obl{\D ||- f}"
  (called a \emph{transformation}) using the following inference rules
  whose leaves are primitive claims.
  %
  \begin{gather*}
    \I[\USE\ \DEFS]{"\sigma.\ \USE\ \Phi\ \DEFS\ \P : \obl{\G ||- e} --> \obl{\D ||- f}"}
      {"\sigma.\ \USE\ \Phi : \obl{\G \USING \P ||- e} --> \obl{\D ||- f}"}
    \\[1ex]
    \I[\HIDE\ \DEFS]{"\sigma.\ \HIDE\ \Phi\ \DEFS\ \P : \obl{\G ||- e} --> \obl{\D \HIDING \P ||- f}"}
       {"\sigma.\ \HIDE\ \Phi : \obl{\G ||- e} --> \obl{\D ||- f}"}
    \\
    \I[\DEFINE]{"\sigma.\ \DEFINE\ o \DEF \delta : \obl{\G ||- e} --> \obl{\G, \hide{o \DEF \delta} ||- e}"}
    \\
    \I["\USE_0"]{"\sigma.\ \USE\ \cdot : \obl{\G ||- e} --> \obl{\G ||- e}"}
    \SP
    \I["\HIDE_0"]{"\sigma.\ \HIDE\ \cdot : \obl{\G ||- e} --> \obl{\G ||- e}"}
    \\[1ex]
    \I["\USE_1"]{"\sigma.\ \USE\ \Phi, \obl{\G_0 ||- e_0} : \obl{\G ||- e} --> \obl{\D, \obl{\G_0 ||- e_0} ||- f}"}
      {"\sigma.\ \USE\ \Phi : \obl{\G ||- e} --> \obl{\D ||- f}"
       &
       "\OBVIOUS : \obl{\D, \G_0 ||- e_0}"
      }
    \\[1ex]
    \I["\HIDE_1"]{"\sigma.\ \HIDE\ \Phi, \phi : \obl{\G_0, \phi, \G_1 ||- e} --> \obl{\D ||- f}"}
      {"\sigma.\ \HIDE\ \Phi : \obl{\G_0, hide{\phi}, \G_1 ||- e} --> \obl{\D ||- f}"}
    \\
    \I["\TAKE_0"]{"\sigma.\ \TAKE\ \cdot : \obl{\G ||- e} --> \obl{\G ||- e}"}
    \LSP
    \I["\WITNESS_0"]{"\sigma.\ \WITNESS\ \cdot : \obl{\G ||- e} --> \obl{\G ||- e}"}
    \\[1ex]
    \I["\TAKE_1"]{"\sigma.\ \TAKE\ u, \vec\beta : \obl{\G ||- \forall x : e} --> \obl{\D ||- f}"}
      {"\sigma.\ \TAKE\ \vec\beta : \obl{\G, \NEW u ||- e [x := u]} --> \obl{\D ||- f}"}
    \\[1ex]
    \I["\TAKE_2"]{"\sigma.\ \TAKE\ u \in T, \vec\beta : \obl{\G ||- \forall x \in S : e} --> \obl{\D ||- f}"}
      {"\OBVIOUS : \obl{\G ||- S \subseteq T}"
       &
       "\sigma.\ \TAKE\ \vec\beta : \obl{\G, \NEW u, u \in T ||- e [x := u]} --> \obl{\D ||- f}"}
    \\[1ex]
    \I["\WITNESS_1"]{"\sigma.\ \WITNESS\ w, \W : \obl{\G ||- \exists x : e} --> \obl{\D ||- f}"}
      {"\sigma.\ \WITNESS\ \W : \obl{\G ||- e [x := w]} --> \obl{\D ||- f}"}
    \\[1ex]
    \I["\WITNESS_2"]{"\sigma.\ \WITNESS\ w \in T, \W : \obl{\G ||- \exists x \in S : e} --> \obl{\D ||- f}"}
      {\begin{array}[b]{c}
          "\OBVIOUS : \obl{\G ||- T \subseteq S}" \\
          "\OBVIOUS : \obl{\G ||- w \in T}"
       \end{array}
       &
       "\sigma.\ \WITNESS\ \W : \obl{\G, w \in T ||- e [x := w]} --> \obl{\D ||- f}"}
    \\[1ex]
    \I[\HAVE]{"\sigma.\ \HAVE\ g : \obl{\G ||- e => f} --> \obl{\G, g ||- f}"}
      {"\OBVIOUS : \obl{\G, e ||- g}"}
    \\[1ex]
    \I[$\ASSERT_1$]{"\s n.\ \obl{\D ||- f}\ \PROOF\ \pi : \obl{\G ||- e} --> \obl{\G, \obl{\D ||- f} ||- e}"}
      {"\pi : \obl{\G, \hide{\lnot e}, \D ||- f}"}
    \\[1ex]
    \I[$\ASSERT_2$]{"\s n l.\ \obl{\D ||- f}\ \PROOF\ \pi : \obl{\G ||- e} --> \obl{\G, \s n l \DEF \obl{\D ||- f}, \hide{\s n l} ||- e}"}
      {"\pi : \obl{\G, \s n l \DEF \obl{\D ||- f}, \hide{\lnot e}, \D ||- f}"}
    \\[1ex]
    \I[\CASE]{"\sigma.\ \CASE\ g\ \PROOF\ \pi : \obl{\G ||- e}
                     --> \obl{\D ||- f}"}
      {"\sigma.\ \obl{g ||- e}\ \PROOF\ \pi : \obl{\G ||- e} --> \obl{\D ||- f}"}
    \\[1ex]
    \I[$\SUFFICES_1$]{"\s n.\ \SUFFICES\ \obl{\D ||- f}\ \PROOF\ \pi : \obl{\G ||- e} --> \obl{\G, \hide{\lnot e}, \D ||- f}"}
      {"\pi : \obl{\G, \obl{\D ||- f} ||- e}"}
    \\[1ex]
    \I[$\SUFFICES_2$]{"\s n l.\ \SUFFICES\ \obl{\D ||- f}\ \PROOF\ \pi : \obl{\G ||- e} --> \obl{\G, \s n l \DEF \obl{\D ||- f}, \hide{\lnot e}, \D ||- f}"}
      {"\pi : \obl{\G, \s n l \DEF \obl{\D ||- f}, \hide{\s n l} ||- e}"}
    \\[1ex]
    \I[\PICK]{"\sigma.\ \PICK\ \vec\beta : p\ \PROOF\ \pi :
                \obl{\G ||- e} --> \obl{\G, \refl{\vec \beta}, p ||- e}"}
      {"\pi : \obl{\G ||- \exists \vec \beta : p}"}
  \end{gather*}
\end{defn}

% \ednote{SM}{I marginally rewrote the rules $\USE\ \DEFS$ and $\HIDE\ \DEFS$
%   to make them easier to read (IMHO). Please check that I didn't blunder.}
%
% I agree with this change. Thanks. -- KC

Both the checking and the transformation derivations are
deterministic: the conclusion of each derivation uniquely determines
the entire derivation. Recall that in section~\ref{sec:obligations} we
defined the \textit{proof obligations} of a claim to be the primitive
claims that jointly imply the assertion of the claim. We now make this
definition formal.

\begin{defn}[Proof Obligations] \label{defn:obligations} \mbox{}
  \begin{ecom}
  \item The \emph{proof obligations} of a claim "\pi : \obl{\G ||- e}"
    are the set of primitive claims at the leaves of its checking
    derivation (defn.~\ref{defn:checking}).
  \item The \emph{proof obligations} of a transformation
    "\sigma.\,\tau : \obl{\G ||- e} --> \obl{\D ||- e}" are the set of
    primitive claims at the leaves of its transformation derivation
    (defn.~\ref{defn:transform}).
  \end{ecom}
\end{defn}

\subsection{Correctness}
\label{apx:correctness}

If the obligations of a claim are true, then the claim itself ought to
be true (theorem~\ref{thm:meaning}). In this section we shall prove
this theorem by analysis of the checking and transformation
derivations. The proof of this theorem will moreover demonstrate how
to compute a proof of the structural obligation of a claim, which is a
(meta-)implication from the obligations of a claim to the assertion of
the claim.

\begin{defn}[Truth of Checking and Transformation] \label{defn:proc-truth} \mbox{}
  %
  \begin{ecom}
  \item The proof obligations "\OBVIOUS : \obl{\G ||- e}" and
    "\OMITTED : \obl{\G ||- e}" are \emph{true} iff "\obl{\G ||- e}"
    is true (defn.~\ref{defn:true})
  \item The claim "\pi : \obl{\G ||- e}" is \emph{true} iff its proof
    obligations are true.
  \item The transformation "\sigma.\,\tau : \obl{\G ||- e} --> \obl{\D
      ||- f}" is \emph{true} iff its proof obligations are true.
  \end{ecom}
\end{defn}

We now prove the correctness of our procedure for generating primitive
claims; Thm.~\ref{thm:meaning} is just assertion~1 of the following
theorem.

\begin{thm}[Correctness] \label{thm:correctness} \mbox{}
  %
  \begin{ecom}
  \item If "\pi : \obl{\G ||- e}" is true, then "\obl{\G ||- e}" is
    true.
  \item If "\sigma.\,\tau : \obl{\G ||- e} --> \obl{\D ||- f}" is true and
    "\obl{\D ||- f}" is true, then "\obl{\G ||- e}" is true.
  \end{ecom}
\end{thm}

\ednote{SM}{Thm.~\ref{thm:meaning} moreover requires that
  "\pi : \obl{\G ||- e}" be a meaningful claim.\\
  Suggestion: if we omitted
  Def.~\ref{defn:proc-truth} and replaced
  ``"\pi : \obl{\G ||- e}" is true'' by ``the
  proof obligations generated by "\pi : \obl{\G ||- e}" are true'',
  introducing the former as a shorthand for the latter in the proof
  (and similarly for assertion~2),
  the correspondence between the assertions of the two theorems would
  be more obvious.}

\ednote{KC}{As far as I can tell, defn.~\ref{defn:proc-truth} does
  exacly what you propose. Since defns.~\ref{defn:checking} and
  \ref{defn:transform} do not specify a ``generation'' of proof
  obligations as such, I say ``the proof obligations of a claim''
  (defn.~\ref{defn:obligations}).

  \medskip I agree about the further requirement that the claim is
  meaningful. I implicitly make use of this assumption in the form of
  the i.h. in, for example, \s{2}{10} below. I shall add it right
  after this commit.}

\begin{proof}
  Let "\DD" be the checking derivation in (1) and let "\EE" be the transformation
  derivation in (2). The proof will be by lexicographic
  induction on the structures of "\DD" and "\EE", with a true
  transformation allowed to justify a true claim.

  \begin{ecom}[{$\s1$}1.]
  \item If "\pi : \obl{\G ||- e}" is true, then "\obl{\G ||-
      e}" is true.
    %
    \begin{ecom}[{$\s2$}1.] \setlength{\itemsep}{6pt}
    \item \Case "\pi" is "\BY\ \Phi\ \DEFS\ \P", \ie,
      \begin{gather*}
        \DD = 
        \Im[\BY.]{"\BY\ \Phi\ \DEFS\ \P : (\G ||- e)"}
           {\deduce{"\s{0}.\ \USE\ \Phi\ \DEFS\ \P" : "(\G ||- e) --> (\D ||- f)"}{\EE_0}
            &
            "\OBVIOUS : (\D ||- f)"}
      \end{gather*}
      \begin{ecom}[{$\s3$}1.]
      \item "\obl{\D ||- f}" is true
        %
        \by defn.~\ref{defn:proc-truth}.
      \item If "\obl{\D ||- f}" then "\obl{\G ||- e}" is true
        %
        \by i.h. (induction hypothesis) on "\EE_0".
      \item \Qed
        %
        \by \s31, \s32.
      \end{ecom}

    \item \Case "\pi" is "\sigma.\, \QED\ \PROOF\ \pi_0", \ie, 
      $
      \DD =
      \Im[\QED.]{"\sigma.\ \QED\ \PROOF\ \pi_0 : (\G ||- e)"}
         {\deduce{"\pi_0 : (\G ||- e)"}{\DD_0}}$
      %
      \by i.h. on "\DD_0".

    \item \Case "\pi" is "\sigma_1.\,\tau_1\ \ \sigma_2.\,\tau_2\ \ \dotsb\ \ \sigma_n.\,\tau_n" with "n > 1", \ie,
      \begin{gather*}
        \DD =
        \Im[non-\QED.]{"\sigma_1.\,\tau_1\ \ \sigma_2.\,\tau_2\ \ \dotsb\ \ \sigma_n.\,\tau_n : (\G ||- e)"}
          {\deduce{"\sigma_1.\,\tau_1 : (\G ||- e) --> (\D ||- f)"}{\EE_0}
           &
           \deduce{"\sigma_2.\,\tau_2\ \dotsb\ \sigma_n.\,\tau_n : (\D ||- f)"}{\DD_0}
          }
      \end{gather*}

      \begin{ecom}[{$\s3$}1.]
      \item [\s31.] "\obl{\D ||- f}" is true
        % 
        \by i.h. on "\DD_0".
      \item [\s32.] If "\obl{\D ||- f}" is true, then "\obl{\G ||- e}"
        is true
        % 
        \by i.h. on "\EE_0".
      \item [\s33.] \Qed
        %
        \by \s31, \s32.
      \end{ecom}
    \item \Qed \by \s21, \s22, \s23.
    \end{ecom}

  \item If "\sigma.\,\tau : \obl{\G ||- e} --> \obl{\D ||- f}" is true
    and "\obl{\D ||- f}" is true, then "\obl{\G ||- e}" is true.
    %
    \begin{ecom}[{$\s2$}1.] \setlength{\itemsep}{6pt}

    \item \Case "\tau" is "\USE\ \Phi\ \DEFS\ \P", \ie,
      \begin{gather*}
        \EE =
        \Im[\USE\ \DEFS.]{"\sigma.\ \USE\ \Phi\ \DEFS\ \P : \obl{\G ||- e} --> \obl{\D ||- f}"}
          {\deduce{"\sigma.\ \USE\ \Phi : \obl{\G \USING \P ||- e} --> \obl{\D ||- f}"}{\EE_0}}
      \end{gather*}

      \begin{ecom}[{$\s3$}1.]
      \item If "\obl{\D ||- f}" is true then "\obl{\G \USING \P ||- e}" is true
        %
        \by i.h. on "\EE_0".
      \item \Qed
        %
        \by \s31, lemma~\ref{thm:use/hide-defs}.
      \end{ecom}

% \ednote{SM}{I edited this step so that it agrees with the rewritten rules.}
%
% OK. -- KC

    \item \Case "\tau" is "\HIDE\ \Phi\ \DEFS\ \P", \ie,
      \begin{gather*}
        \EE =
        \Im[\HIDE\ \DEFS.]{"\sigma.\ \HIDE\ \Phi\ \DEFS\ \P : \obl{\G ||- e} --> \obl{\D \HIDING \P ||- f}"}
           {\deduce{"\sigma.\ \HIDE\ \Phi : \obl{\G ||- e} --> \obl{\D ||- f}"}{\EE_0}}
      \end{gather*}

      \begin{ecom}[{$\s3$}1.]
      \item If "\obl{\D ||- f}" is true then "\obl{\G ||- e}" is true
        %
        \by i.h. on "\EE_0".
      \item If "\obl{\D \HIDING \P ||- f}" is true then "\obl{\D ||- f}" is true
        %
        \by lemma~\ref{thm:use/hide-defs}.
      \item \Qed
        %
        \by \s31, \s32.
      \end{ecom}

\ednote{SM}{I edited this step so that it agrees with the rewritten rules.}

    \item \Case "\tau" is "\DEFINE\ o \DEF \delta", \ie, 
      \begin{gather*}
        \EE =
        \Im[\DEFINE.]{"\sigma.\ \DEFINE\ o \DEF \delta : \obl{\G ||- e} --> \obl{\G, \hide{o \DEF \delta} ||- e}"}{}
      \end{gather*}

      \begin{ecom}[{$\s3$}1.]
      \item "o" is not free in "e"
        %
        \by \tlatwo syntactic restrictions.
\ednote{SM}{This comes as somewhat of a surprise.}
\ednote{KC}{I agree. Strictly speaking, the \PM does not enforce the \tlaplus bound variable freshness restriction. It avoids inadvertent capture by shifting (using de Bruijn terminology) "e" by one in the result of the transformation. It seemed best to me to avoid going down this explanatory detour, but suggestions for improvement will be gratefully accepted.}
      \item \Qed
        %
        \by \s21, lemma~\ref{thm:delete}.
      \end{ecom}

    \item \Case "\tau" is "\USE\ \cdot", \ie,
      $
      \EE = 
      \Im["\USE_0".]{"\sigma.\ \USE\ \cdot : \obl{\G ||- e} --> \obl{\G ||- e}"}{}
      $
      \Trivial

    \item \Case "\tau" is "\HIDE\ \cdot", \ie,
      $
      \EE = 
      \Im["\HIDE_0".]{"\sigma.\ \HIDE\ \cdot : \obl{\G ||- e} --> \obl{\G ||- e}"}{}
      $
      \Trivial

    \item \Case "\tau" is "\USE\ \Phi, \phi", \ie,
      \begin{gather*}
        \EE = 
        \Im["\USE_1"]{"\sigma.\ \USE\ \Phi, \obl{\G_0 ||- e_0} : \obl{\G ||- e} --> \obl{\D_0, \obl{\G_0 ||- e_0} ||- f}"}
          {\deduce{"\sigma.\ \USE\ \Phi : \obl{\G ||- e} --> \obl{\D_0 ||- f}"}{\EE_0}
           &
           "\OBVIOUS : \obl{\D_0, \G_0 ||- e_0}"
          }      
      \end{gather*}

      \begin{ecom}[{$\s3$}1.]
      \item "\obl{\D_0, \G_0 ||- e_0}" is true
        %
        \by defn.~\ref{defn:proc-truth}.
      \item "\obl{\D_0 ||- f}" is true
        %
        \by \s31, cut (fact~\ref{thm:cut}).
      \item If "\obl{\D_0 ||- f}" is true, then "\obl{\G ||- e}" is true
        %
        \by i.h. on "\EE_0".
      \item \Qed
        %
        \by \s32, \s33.
      \end{ecom}

    \item \Case "\tau" is "\HIDE\ \Phi, \phi", \ie,
      \begin{gather*}
        \EE =
        \Im["\HIDE_1".]{"\sigma.\ \HIDE\ \Phi, \phi : \obl{\G_0, \phi, \G_1 ||- e} --> \obl{\D ||- f}"}
           {\deduce{"\sigma.\ \HIDE\ \Phi : \obl{\G_0, hide{\phi}, \G_1 ||- e} --> \obl{\D ||- f}"}{\EE_0}}
      \end{gather*}

      \begin{ecom}[{$\s3$}1.]
      \item If "\obl{\D ||- f}" is true, then "\obl{\G_0, \hide{\phi}, \G_1 ||- e}" is true
        %
        \by i.h. on "\EE_0".
      \item If "\obl{\G_0, \hide{\phi}, \G_1 ||- e}" is true, then
        "\obl{\G_0, \phi, \G_1 ||- e}" is true
        %
        \by defn.~\ref{defn:true}, weakening (fact~\ref{thm:weaken}).
      \item \Qed
        %
        \by \s31, \s32.
      \end{ecom}

    \item \Case "\tau" is "\TAKE\ \cdot", \ie,
      $
      \EE = 
      \Im["\TAKE_0".]{"\sigma.\ \TAKE\ \cdot : \obl{\G ||- e} --> \obl{\G ||- e}"}{}
      $
      \Trivial

    \item \Case "\tau" is "\WITNESS\ \cdot", \ie,
      $
      \EE = 
      \Im["\WITNESS_0".]{"\sigma.\ \WITNESS\ \cdot : \obl{\G ||- e} --> \obl{\G ||- e}"}{}
      $
      \Trivial

    \item \Case "\tau" is "\TAKE\ u, \vec \beta", \ie,
      \begin{gather*}
        \EE =
        \Im["\TAKE_1".]{"\sigma.\ \TAKE\ u, \vec\beta : \obl{\G ||- \forall x : e} --> \obl{\D ||- f}"}
           {\deduce{"\sigma.\ \TAKE\ \vec\beta : \obl{\G, \NEW u ||- e [x := u]} --> \obl{\D ||- f}"}{\EE_0}}
      \end{gather*}

      \begin{ecom}[{$\s3$}1.]
      \item "\obl{\G, \NEW u ||- e [x := u]}" is true
        %
        \by i.h. on "\EE_0".
      \item \Qed
        %
        \by \s31{} and predicate logic.
% original version
%        \by \s32.
      \end{ecom}
% \ednote{SM}{I changed the justification of the QED step, which didn't
%   make sense. Kaustuv, please check.}
%
% OK, I agree. -- KC

    \item \Case "\tau" is "\sigma.\ \TAKE\ u \in T", \ie,
      \begin{gather*}
        \EE =
        \Im["\TAKE_2".]{"\sigma.\ \TAKE\ u \in T, \vec\beta : \obl{\G ||- \forall x \in S : e} --> \obl{\D ||- f}"}
           {"\OBVIOUS : \obl{\G ||- S \subseteq T}"
            &
            \deduce{"\sigma.\ \TAKE\ \vec\beta : \obl{\G, \NEW u, u \in T ||- e [x := u]} --> \obl{\D ||- f}"}{\EE_0}}
      \end{gather*}

      \begin{ecom}[{$\s3$}1.]
      \item "\obl{\G, \NEW u, u \in T ||- e [x := u]}" is true
        %
        \by i.h on "\EE_0".
      \item "\obl{\G, \NEW u, u \in S ||- u \in T}" is true
        \begin{ecom}[{$\s4$}1.]
        \item "\obl{\G, \NEW u ||- S \subseteq T}" is true
          %
          \by defn.~\ref{defn:proc-truth}, weakening
          (fact~\ref{thm:weaken}).
        \item \Qed
          %
          \by \s41, defn. of "\subseteq".
        \end{ecom}
      \item "\obl{\G, \NEW u, u \in S ||- e[x := u]}" is true
        %
        \by \s31, \s32, cut (fact~\ref{thm:cut}).
      \item \Qed
        %
        \by \s33{} and predicate logic.
      \end{ecom}
% \ednote{SM}{I added ``and predicate logic'': if you edit, please keep
%   it consistent with the preceding step.}
%
% No need to edit it. I agree with your modification. -- KC

    \item \Case "\tau" is "\WITNESS\ w, \W", \ie,
      \begin{gather*}
        \EE =
        \Im["\WITNESS_1".]{"\sigma.\ \WITNESS\ w, \W : \obl{\G ||- \exists x : e} --> \obl{\D ||- f}"}
           {\deduce{"\sigma.\ \WITNESS\ \W : \obl{\G ||- e [x := w]} --> \obl{\D ||- f}"}{\EE_0}}
      \end{gather*}

      \begin{ecom}[{$\s3$}1.]
      \item "\obl{\G ||- e [x := w]}" is true
        %
        \by i.h. on "\EE_0".
      \item \Qed
        %
        \by \s31.
      \end{ecom}

    \item \Case "\tau" is "\WITNESS\ w \in T, \W" and:
      \begin{gather*}
        \EE =
        \Im["\WITNESS_2".]
           {"\sigma.\ \WITNESS\ w \in T, \W : \obl{\G ||- \exists x \in S : e} --> \obl{\D ||- f}"}
           {\begin{array}[b]{c}
               "\OBVIOUS : \obl{\G ||- T \subseteq S}" \\
               "\OBVIOUS : \obl{\G ||- w \in T}"
            \end{array}
            &
            \deduce{"\sigma.\ \WITNESS\ \W : \obl{\G, w \in T ||- e [x := w]} --> \obl{\D ||- f}"}{\EE_0}}
     \end{gather*}

     \begin{ecom}[{$\s3$}1.]
     \item "\obl{\G, w \in T ||- e [x := w]}" is true
       % 
       \by i.h. on "\EE_0".
     \item "\obl{\G ||- w \in T}" is true
       % 
       \by defn.~\ref{defn:proc-truth}.
     \item "\obl{\G ||- e [x := w]}" is true
       % 
       \by \s31, \s32, cut (fact~\ref{thm:cut}).
     \item "\obl{\G ||- w \in S}" is true
       \begin{ecom}[{$\s4$}1.]
       \item "\obl{\G, w \in T ||- w \in S}" is true
         % 
         \by defn.~\ref{defn:proc-truth}, defn. of "\subseteq".
       \item \Qed
         % 
         \by \s41, \s32, cut (fact~\ref{thm:cut}).
       \end{ecom}
     \item \Qed
       % 
       \by \s33, \s34, and predicate logic.
     \end{ecom}
     
    \item "\tau" is "\HAVE\ g", \ie,
      \begin{gather*}
        \EE =
        \Im[\HAVE.]{"\sigma.\ \HAVE\ g : \obl{\G ||- e => f} --> \obl{\G, g ||- f}"}
           {"\OBVIOUS : \obl{\G, e ||- g}"}
      \end{gather*}

      \begin{ecom}[{$\s3$}1.]
      \item "\obl{\G, e, g ||- f}" is true
        %
        \by weakening (fact~\ref{thm:weaken}).
      \item "\obl{\G, e ||- g}" is true
        %
        \by defn.~\ref{defn:proc-truth}.
      \item "\obl{\G, e ||- f}" is true
        %
        \by \s31, \s32, cut (fact~\ref{thm:cut}).
      \item "\obl{\G ||- e => f}" is true
        %
        \by \s33.
      \end{ecom}

    \item "\sigma.\,\tau" is "\s n.\ \obl{\W ||- g}\ \PROOF\ \pi", \ie, 
      \begin{gather*}
        \EE =
        \Im[$\ASSERT_1$.]{"\s n.\ \obl{\W ||- g}\ \PROOF\ \pi : \obl{\G ||- e} --> \obl{\G, \obl{\W ||- g} ||- e}"}
           {\deduce{"\pi : \obl{\G, \hide{\lnot e}, \W ||- g}"}{\DD_0}}
      \end{gather*}

      \begin{ecom}[{$\s3$}1.]
      \item "\obl{\G, \hide{\lnot e}, \obl{\W ||- g} ||- e}" is true
        %
        \by weakening (fact~\ref{thm:weaken}).
      \item "\obl{\G, \hide{\lnot e}, \W ||- g}" is true
        %
        \by i.h. on "\DD_0".
      \item "\obl{\G, \hide{\lnot e} ||- e}" is true
        %
        \by \s31, \s32, cut (fact~\ref{thm:cut}).
      \item \Qed
        %
        \by \s33.
      \end{ecom}

    \item \Case "\sigma.\,\tau" is "\s n l.\ \obl{\W ||- g}\ \PROOF\ \pi", \ie,
      \begin{gather*}
        \EE =
        \Im[$\ASSERT_2$.]{"\s n l.\ \obl{\W ||- g}\ \PROOF\ \pi : \obl{\G ||- e} --> \obl{\G, \s n l \DEF \obl{\W ||- g}, \hide{\s n l} ||- e}"}
           {\deduce{"\pi : \obl{\G, \s n l \DEF \obl{\W ||- g}, \hide{\lnot e}, \W ||- g}"}{\DD_0}}
      \end{gather*}

      \begin{ecom}[{$\s3$}1.]
      \item "\obl{\G, \s n l \DEF \obl{\W ||- g}, \hide{\lnot e}, \hide{\s n l}, \W ||- g}" is true
        % 
        \by i.h. on "\DD_0", weakening (fact~\ref{thm:weaken}).
      \item "\obl{\G, \s n l \DEF \obl{\W ||- g}, \hide{\lnot e}, \hide{\s n l}, \obl{\W ||- g} ||- e}" is true
        % 
        \by weakening.
      \item "\obl{\G, \s n l \DEF \obl{\W ||- g}, \hide{\lnot e}, \hide{\s n l} ||- e}" is true
        % 
        \by \s31, \s32, cut (fact~\ref{thm:cut}).
      \item "\isa{\G, \s n l \DEF \obl{\W ||- g}, \hide{\lnot e}, \hide{\s n l} ||- e} = \isa{\G, \s n l \DEF \obl{\W ||- g} ||- e}"
        % 
        \by defn.~\ref{defn:isabelle-embedding}.
      \item "\obl{\G, \s n l \DEF \obl{\W ||- g} ||- e}" is true
        % 
        \by \s33, \s34, defn.~\ref{defn:true}.
      \item \Qed
        % 
        \by \s35, strengthening (fact~\ref{thm:delete}).
% SM: was "\by \s34, strengthening (fact~\ref{thm:delete})."
      \end{ecom}
      
    \item "\tau" is "\CASE\ g\ \PROOF\ \pi", \ie,
      \begin{gather*}
        \EE =
        \Im[\CASE.]{"\sigma.\ \CASE\ g\ \PROOF\ \pi : \obl{\G ||- e} --> \obl{\D ||- f}"}
           {\deduce{"\sigma.\ \obl{g ||- e}\ \PROOF\ \pi : \obl{\G ||- e} --> \obl{\D ||- f}"}{\EE_0}}
      \end{gather*}

      \textit{By} i.h. on "\EE_0".

    \item "\tau" is "\s n.\ \SUFFICES\ \obl{\W ||- g}\ \PROOF\ \pi", \ie,
      \begin{gather*}
        \EE =
        \Im[$\SUFFICES_1$.]{"\s n.\ \SUFFICES\ \obl{\D ||- f}\ \PROOF\ \pi : \obl{\G ||- e} --> \obl{\G, \hide{\lnot e}, \W ||- g}"}
           {\deduce{"\pi : \obl{\G, \obl{\W ||- g} ||- e}"}{\DD_0}}
      \end{gather*}

    \begin{ecom}[{$\s3$}1.]
      \item "\obl{\G, \hide{\lnot e}, \obl{\W ||- g} ||- e}" is true
        %
        \by i.h. on "\DD_0", weakening (fact~\ref{thm:weaken}).
      \item "\obl{\G, \hide{\lnot e} ||- e}" is true
        %
        \by \s31, cut (fact~\ref{thm:cut}).
      \item \Qed
        %
        \by \s32.
      \end{ecom}

    \item "\sigma.\,\tau" is "\s n l.\ \SUFFICES\ \obl{\W ||- g}\ \PROOF\ \pi", \ie,
      \begin{gather*}
        \EE =
        \Im[$\SUFFICES_2$.]{"\s n l.\ \SUFFICES\ \obl{\W ||- g}\ \PROOF\ \pi : \obl{\G ||- e} --> \obl{\G, \s n l \DEF \obl{\W ||- g}, \hide{\lnot e}, \W ||- g}"}
           {\deduce{"\pi : \obl{\G, \s n l \DEF \obl{\W ||- g}, \hide{\s n l} ||- e}"}{\DD_0}}
      \end{gather*}

      \begin{ecom}[{$\s3$}1.]
      \item "\obl{\G, \s n l \DEF \obl{\W ||- g}, \hide{\s n l} ||- e}" is true
        %
        \by i.h. on "\DD_0".
      \item "\isa{\G, \s n l \DEF \obl{\W ||- g}, \hide{\s n l} ||- e} = \isa{\G, \s n l \DEF \obl{\W ||- g} ||- e}"
        %
        \by defn.~\ref{defn:isabelle-embedding}.
      \item "\obl{\G, \s n l \DEF \obl{\W ||- g} ||- e}" is true
        %
        \by \s31, \s32, defn.~\ref{defn:true}.
      \item \Qed
        % 
        \by \s33, strenghening (fact~\ref{thm:delete}).
      \end{ecom}

    \item \Case "\tau" is "\PICK\ \vec\beta : p\ \PROOF\ \pi", \ie,
      \begin{gather*}
        \EE =
        \Im[\PICK.]{"\sigma.\ \PICK\ \vec\beta : p\ \PROOF\ \pi :
                        \obl{\G ||- e} --> \obl{\G, \refl{\vec \beta}, p ||- e}"}
           {\deduce{"\pi : \obl{\G ||- \exists \vec \beta : p}"}{\DD_0}}
      \end{gather*}

      \begin{ecom}[{$\s3$}1.]
      \item "\obl{\G, \exists \vec \beta : p ||- e}" is true
        %
        \by predicate logic.
% original version:
%        \Trivial.
      \item "\obl{\G ||- \exists \vec \beta : p}" is true
        %
        \by i.h. on "\DD_0".
      \item \Qed
        %
        \by \s31, \s32, cut (fact~\ref{thm:cut}).
      \end{ecom}

    \item \Qed
      %
      \by \s21, \dots, \s220
    \end{ecom}
  \item \Qed
    %
    \by \s11, \s12.
  \end{ecom}
\end{proof}

% \noindent
% Note that theorem~\ref{thm:meaning} in sec.~\ref{sec:obligations} is
% case 1 of the above correctness theorem.

The rules "\ASSERT_1", "\ASSERT_2", "\SUFFICES_1" and "\SUFFICES_2"
for transformation derivations, when transforming "\obl{\G ||- e}",
add the negation of the conclusion "e" into the context for the proof
of the assertion in the step. This is required because the logic of
\tlatwo is classical. For instance, consider the following proof of
double-negation elimination (where "A" is a constant proposition).
%
\begin{quote}
  \begin{tabbing}
    \THEOREM \= \ASSUME "\lnot \lnot A" \PROVE "A" \\
    \s11.\ \= \FALSE \BY "\lnot A", "\lnot \lnot A" \\
    \s22.\ \> \QED \BY \s11
  \end{tabbing}
\end{quote}
%
Intuitionistically, we can prove "\FALSE" from "\lnot A" and "\lnot
\lnot A", and "A" from "\FALSE"; moreover "\lnot \lnot A" is present
in the context, so \BY can certainly appeal to it. Therefore the only
classical feature in this proof is the fragment "\BY\ \lnot A". If
"\lnot A" were not added to the context before checking the subproof
of step \s11, the obligation produced for "\BY\ \lnot A" would not be
true.

\ednote{SM}{Kaustuv, there is one thing that troubles me here. The
  negated conclusion is added as a hidden assumption, and you say that
  an assertion is true is its Isabelle translation is. However, hidden
  assumptions are left out during the translation to Isabelle (and you
  even use this in the justifications of the \ASSERT{} rules). So how
  is Isabelle going to justify the proof obligation for $\lnot A$?}

\ednote{KC}{Thanks for spotting this. The obligations generated for the \BY steps unhide all
  definitions and facts in the context. That is, the "\USE_1" step should be as follows:
  \begin{gather*}
    \I["\USE_1"]{"\sigma.\ \USE\ \Phi, \obl{\G_0 ||- e_0} : \obl{\G ||- e} --> \obl{\D, \obl{\G_0 ||- e_0} ||- f}"}
      {"\sigma.\ \USE\ \Phi : \obl{\G ||- e} --> \obl{\D ||- f}"
       &
       "\OBVIOUS : \obl{\D, \G_0 ||- e_0}_u"
      }
  \end{gather*}
  where
  \begin{gather*}
    \obl{"\G ||- e"}_u = "\obl{\G}_u ||- e" \\
    \begin{aligned}
      \obl{\cdot}_u &= \cdot &
      \obl{\G, \NEW x}_u &= \obl{\G}_u, \NEW x \\
      \obl{\G, o \DEF \delta}_u &= \obl{\G}_u, o \DEF \delta &
      \obl{\G, \phi}_u &= \obl{\G}_u, \phi \\
      \obl{\G, \hide{o \DEF \delta}}_u &= \obl{\G}_u, o \DEF \delta &
      \obl{\G, \hide{\phi}}_u &= \obl{\G}_u, \phi. \\
    \end{aligned}
  \end{gather*}

  Not excusing my mistake, but such a definition used to be present in
  the appendix at one point, but I appear to have inadvertently
  skipped it when porting the rules over over to the new
  terminology. Will add it after this commit.}

%%%% PLEASE DO NOT EDIT BELOW THIS LINE
\ifx\master\undefined
{\let\master\relax%%% -*- mode: LaTeX; TeX-master: "main.tex"; -*-

\ifx\master\undefined
\documentclass[a4paper]{easychair}
\usepackage{submission}
\begin{document}
{\let\master\relax %%% -*- mode: LaTeX; TeX-master: "main.tex"; -*-

\ifx\master\undefined
\documentclass[a4paper]{easychair}
\usepackage{submission}
\begin{document}
\fi
%%%% PLEASE DO NOT EDIT ABOVE THIS LINE

\title{A \tlaplus Proof System}

\titlerunning{A \tlaplus Proof System}

% \volumeinfo
% 	{P. Rudnicki, G. Sutcliffe} % editors
% 	{2}                         % number of editors
% 	{KEAPPA 2008}               % event
% 	{1}                         % volume
% 	{1}                         % issue
% 	{1}                         % starting page number


%  Alphabetically by surname
\author{
  Kaustuv Chaudhuri \\
  INRIA \\
  \and
  Damien Doligez \\
  INRIA \\
  \and
  Leslie Lamport \\
  Microsoft Research \\
  \and
  Stephan Merz \\
  INRIA \& Loria
}

\authorrunning{Chaudhuri, Doligez, Lamport, and Merz}

\maketitle

%%%% PLEASE DO NOT EDIT BELOW THIS LINE
\ifx\master\undefined
{\let\master\relax %%% -*- mode: LaTeX; TeX-master: "main.tex"; -*-

\ifx\master\undefined
\documentclass[a4paper]{easychair}
\usepackage{submission}
\begin{document}
{\let\master\relax \input{frontmatter}}
\fi
%%%% PLEASE DO NOT EDIT ABOVE THIS LINE

\bibliographystyle{plain}
\bibliography{submission}

%%%% PLEASE DO NOT EDIT BELOW THIS LINE
\ifx\master\undefined
\end{document}
\fi

% LocalWords:  tex Paxos
}
\end{document}
\fi

% LocalWords:  tex Rudnicki Sutcliffe KEAPPA Kaustuv Chaudhuri INRIA Doligez
% LocalWords:  Merz Loria
}
\fi
%%%% PLEASE DO NOT EDIT ABOVE THIS LINE

\bibliographystyle{plain}
\bibliography{submission}

%%%% PLEASE DO NOT EDIT BELOW THIS LINE
\ifx\master\undefined
\end{document}
\fi

% LocalWords:  tex Paxos
}
\end{document}
\fi
