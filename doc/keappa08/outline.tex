%%% -*- mode: LaTeX; TeX-master: "main.tex"; -*-

\ifx\master\undefined
\documentclass[a4paper]{easychair}
\usepackage{submission}
\begin{document}
{\let\master\relax%%% -*- mode: LaTeX; TeX-master: "main.tex"; -*-

\ifx\master\undefined
\documentclass[a4paper]{easychair}
\usepackage{submission}
\begin{document}
\fi
%%%% PLEASE DO NOT EDIT ABOVE THIS LINE

\title{A \tlaplus Proof System}

\titlerunning{A \tlaplus Proof System}

% \volumeinfo
% 	{P. Rudnicki, G. Sutcliffe} % editors
% 	{2}                         % number of editors
% 	{KEAPPA 2008}               % event
% 	{1}                         % volume
% 	{1}                         % issue
% 	{1}                         % starting page number


%  Alphabetically by surname
\author{
  Kaustuv Chaudhuri \\
  INRIA \\
  \and
  Damien Doligez \\
  INRIA \\
  \and
  Leslie Lamport \\
  Microsoft Research \\
  \and
  Stephan Merz \\
  INRIA \& Loria
}

\authorrunning{Chaudhuri, Doligez, Lamport, and Merz}

\maketitle

%%%% PLEASE DO NOT EDIT BELOW THIS LINE
\ifx\master\undefined
{\let\master\relax %%% -*- mode: LaTeX; TeX-master: "main.tex"; -*-

\ifx\master\undefined
\documentclass[a4paper]{easychair}
\usepackage{submission}
\begin{document}
{\let\master\relax %%% -*- mode: LaTeX; TeX-master: "main.tex"; -*-

\ifx\master\undefined
\documentclass[a4paper]{easychair}
\usepackage{submission}
\begin{document}
\fi
%%%% PLEASE DO NOT EDIT ABOVE THIS LINE

\title{A \tlaplus Proof System}

\titlerunning{A \tlaplus Proof System}

% \volumeinfo
% 	{P. Rudnicki, G. Sutcliffe} % editors
% 	{2}                         % number of editors
% 	{KEAPPA 2008}               % event
% 	{1}                         % volume
% 	{1}                         % issue
% 	{1}                         % starting page number


%  Alphabetically by surname
\author{
  Kaustuv Chaudhuri \\
  INRIA \\
  \and
  Damien Doligez \\
  INRIA \\
  \and
  Leslie Lamport \\
  Microsoft Research \\
  \and
  Stephan Merz \\
  INRIA \& Loria
}

\authorrunning{Chaudhuri, Doligez, Lamport, and Merz}

\maketitle

%%%% PLEASE DO NOT EDIT BELOW THIS LINE
\ifx\master\undefined
{\let\master\relax \input{rearmatter}}
\end{document}
\fi

% LocalWords:  tex Rudnicki Sutcliffe KEAPPA Kaustuv Chaudhuri INRIA Doligez
% LocalWords:  Merz Loria
}
\fi
%%%% PLEASE DO NOT EDIT ABOVE THIS LINE

\bibliographystyle{plain}
\bibliography{submission}

%%%% PLEASE DO NOT EDIT BELOW THIS LINE
\ifx\master\undefined
\end{document}
\fi

% LocalWords:  tex Paxos
}
\end{document}
\fi

% LocalWords:  tex Rudnicki Sutcliffe KEAPPA Kaustuv Chaudhuri INRIA Doligez
% LocalWords:  Merz Loria
}
\fi
%%%% PLEASE DO NOT EDIT ABOVE THIS LINE

\section{Outline}
\label{sec:section-key}

\textbf{NOTE}: This chapter will be removed in the final submitted version.

\paragraph{Quoth Stephan} c. 2008-08-31
\begin{quote}
  Here is a rough sketch of my view of what the outline could be. I
  think the focus should be on pragmatics / implementation issues.

  \begin{itemize}
  \item Introduction explaining the goal of writing hierarchic proofs
    that allow a human to decompose the overall verification effort
    (which we assume to be outside the scope of automated theorem
    provers) -- about 1 page
  \item Outline of the proof language, probably without giving a full
    grammar -- 1.5 pages + 1 page for a figure that shows a complete
    \tlaplus proof (would Cantor's theorem serve as a running
    example?) and that is referred to in the other sections
  \item Describe the verification conditions that are generated by the
    PM for some representative constructs of the language -- 2 pages
  \item Use of Isabelle's automatic proof methods to resolve VCs (in
    particular explain how USEd facts are fed into these methods -- this
    requires some more work and experimentation) -- 1.5 pages
  \item Use of Zenon and in particular proof reconstruction for
    Isabelle, which certifies the proof -- 2 pages
  \end{itemize}

  Add the usual paraphernalia, and we are approximately at the limit
  of 10 pages.
\end{quote}

\paragraph{Quoth Leslie} c. 2008-09-01
\begin{quote}
  Stephan's outline looks good to me.  I would suggest only one minor
  omission.  Zenon's ``proof reconstruction for Isabelle'' seems to be
  part of how Zenon works, and I think the paper should talk about how
  Isabelle \& Zenon are used, not about how they work.  Also, while
  2.5 pages for the proof language seems about right, I suspect that
  devoting 1 page to a complete proof is not the best approach.  I
  suggest short snippets of proof that can be referred to in the other
  sections.

  Here is my minor revision of Stephan's outline--mostly to assign
  numbers to the sections for convenient reference.

  \begin{enumerate}
  \item Introduction explaining the goal of writing hierarchic proofs
    that allow a human to decompose the overall verification effort
    (which we assume to be outside the scope of automated theorem
    provers) 1 page

  \item Outline of the proof language, (definitely without a full
    grammar), with examples referred to later.  2.5 pages

  \item Describe the verification conditions that are generated by the
    PM for some representative constructs of the language 2 pages

  \item Use of Isabelle's automatic proof methods to resolve VCs (in
    particular explain how USEd facts are fed into these methods --
    this requires some more work and experimentation) 1.5 pages

  \item Use of Zenon 1.5 pages
  \end{enumerate}

  I'll be able to write the first draft of sections 1 and 2.  However,
  I'd prefer to wait until we have drafts of sections 3-5 so we'll
  know what examples section 2 needs to include and what section 1 is
  introducing.  It seems to me that Kaustuv should write the first
  draft of section 3.  It looks like sections 4 and 5 should be
  written by Stephan and Damien, respectively, in cooperation with
  Kaustuv.

  Does this sound reasonable?
\end{quote}

%%%% PLEASE DO NOT EDIT BELOW THIS LINE
\ifx\master\undefined
{\let\master\relax%%% -*- mode: LaTeX; TeX-master: "main.tex"; -*-

\ifx\master\undefined
\documentclass[a4paper]{easychair}
\usepackage{submission}
\begin{document}
{\let\master\relax %%% -*- mode: LaTeX; TeX-master: "main.tex"; -*-

\ifx\master\undefined
\documentclass[a4paper]{easychair}
\usepackage{submission}
\begin{document}
\fi
%%%% PLEASE DO NOT EDIT ABOVE THIS LINE

\title{A \tlaplus Proof System}

\titlerunning{A \tlaplus Proof System}

% \volumeinfo
% 	{P. Rudnicki, G. Sutcliffe} % editors
% 	{2}                         % number of editors
% 	{KEAPPA 2008}               % event
% 	{1}                         % volume
% 	{1}                         % issue
% 	{1}                         % starting page number


%  Alphabetically by surname
\author{
  Kaustuv Chaudhuri \\
  INRIA \\
  \and
  Damien Doligez \\
  INRIA \\
  \and
  Leslie Lamport \\
  Microsoft Research \\
  \and
  Stephan Merz \\
  INRIA \& Loria
}

\authorrunning{Chaudhuri, Doligez, Lamport, and Merz}

\maketitle

%%%% PLEASE DO NOT EDIT BELOW THIS LINE
\ifx\master\undefined
{\let\master\relax %%% -*- mode: LaTeX; TeX-master: "main.tex"; -*-

\ifx\master\undefined
\documentclass[a4paper]{easychair}
\usepackage{submission}
\begin{document}
{\let\master\relax \input{frontmatter}}
\fi
%%%% PLEASE DO NOT EDIT ABOVE THIS LINE

\bibliographystyle{plain}
\bibliography{submission}

%%%% PLEASE DO NOT EDIT BELOW THIS LINE
\ifx\master\undefined
\end{document}
\fi

% LocalWords:  tex Paxos
}
\end{document}
\fi

% LocalWords:  tex Rudnicki Sutcliffe KEAPPA Kaustuv Chaudhuri INRIA Doligez
% LocalWords:  Merz Loria
}
\fi
%%%% PLEASE DO NOT EDIT ABOVE THIS LINE

\bibliographystyle{plain}
\bibliography{submission}

%%%% PLEASE DO NOT EDIT BELOW THIS LINE
\ifx\master\undefined
\end{document}
\fi

% LocalWords:  tex Paxos
}
\end{document}
\fi
